\documentclass[12pt, letterpaper]{article}
\usepackage[utf8]{inputenc}
\usepackage{fancyhdr}
\usepackage{listings}
\usepackage{xcolor}
\usepackage{hyperref}

\definecolor{codegreen}{rgb}{0,0.6,0}
\definecolor{codegray}{rgb}{0.5,0.5,0.5}
\definecolor{codepurple}{rgb}{0.58,0,0.82}
\definecolor{backcolour}{rgb}{0.95,0.95,0.92}

\hypersetup{
    colorlinks=true,
    linkcolor=blue,
    filecolor=magenta,
    urlcolor=cyan,
    pdftitle={Sharelatex Example},
    bookmarks=true,
    pdfpagemode=FullScreen,
}

\lstdefinestyle{mystyle}{
    backgroundcolor=\color{backcolour},
    commentstyle=\color{codegreen},
    keywordstyle=\color{magenta},
    numberstyle=\tiny\color{codegray},
    stringstyle=\color{codepurple},
    basicstyle=\ttfamily\footnotesize,
    breakatwhitespace=false,
    breaklines=true,
    captionpos=b,
    keepspaces=true,
    numbers=left,
    numbersep=5pt,
    showspaces=false,
    showstringspaces=false,
    showtabs=false,
    tabsize=2
}

\lstset{style=mystyle}

\title{Sistemas Operativos}
\author{David Lopez}
\date

\pagestyle{fancy}
\fancyhf{}
\rhead{David Lopez}
\lhead{Sistemas operativos}
\rfoot{\thepage}

\begin{document}

\begin{titlepage}
	\clearpage\maketitle
	\thispagestyle{empty}
\end{titlepage}

\tableofcontents{}

\section{conceptos basicos}

Un sistema operativo no es mas que un progama.

\subsection{Funciones del s.o}
Este programa, no obstante, es bastante caracteristico, pues se encarga de \textbf{gestionar los recursos hardware}, ofrece a los demas programas una api de \textbf{llamadas al sistema} y proporciona \textbf{mecanismos de interaccion con el usuario} como el shell.\par

Es necesario gestionar los recursos hardware por que podemos tener varios programas ejecutandose a la vez, y se hace mas importante con los sistemas multiusuario. Los programas competiran por los recursos, por lo que sera necesario administrarlos correctamente.\par

Sobre las llamadas al sistema habra algunos ejercicios para familiarizarnos, hay que tener en cuenta, que a diferencia de un curso de desarrollo sobre el kernel de un sistema operativo, aqui nos centraremos en programas de espacio de usuario y nos limitaremos, al menos por el momento a invocar a estas llamadas, no a implementarlas en incluirlas en el s.o , cosa que tambien se puede hacer.

\subsection{partes principales del s.o}

Podemos distinguir tres partes principales: \textbf{nucleo o kernel, servicios e interfaz de usuario}

\subsection{clasificacion segun su estructura}

\begin{itemize}
	\item{Monolitico (como linux): Tiene todos los componentes integrados en un mismo programa}
	\item{Por capas: cada capa ofrece una interfaz a la capa siguiente}
	\item{Cliente-Servidor (microkernel): el kernel se reduce a la minima expresion y el resto de utilidades se desarrollan en proceso de usuario}
\end{itemize}

\subsection{Como lo usamos ?}

Podemos poner el sistema operativo a trabajar, deliberadamente y para que haga la funcion que queremos, escribiendo un programa que realize llamadas al sistema, las interrupciones de los perifericos y los errores tambien provocara la activacion del s.o

\subsection{ejemplo de programa que invoca llamadas al sistema}
\textit{escribir una funcion en C, que actue como el comando cat de unix, es decir, que imprima el contenido del archivo por la salida estandar}

\lstinputlisting[language=C]{cat.c}

En este codigo, las funciones \textit{open, read, write y close} son llamadas al sistema, cuyo man se puede encontrar en: \href{http://man7.org/linux/man-pages/man2/}{manual de llamadas al sistema}, es fundamental saber utilizar esta herramienta y manejar con solvencia las llamadas al sistema mas habituales.

\section{Procesos}

\subsection{Conceptos basicos}
Un proceso es la unidad de procesamiento gestionada por el sistema operativo.\par
El sistema operativo mantiene una tabla de procesos. En esta tabla se almacena un \textbf{bloque de control de proceso} \textit{(estructura \href{https://github.com/torvalds/linux/blob/master/include/linux/sched.h}{task\_struct} en linux)} por cada proceso.
El BCP contiene la siguiente informacion:
\begin{itemize}
	\item{Identifiacion: PID, USER y relaciones padre-hijo}
	\item{Estado del procesador}
	\item{Informacion de control del proceso}
	\item{Memoria asignada}
	\item{Recursos asignados}
\end{itemize}

\subsection{threads}

Un proceso puede contener varios threads, o hilos de ejecucion que comparten algunos recursos e informacion entre ellos.\par

Cada thread se define como una funcion que se puede lanzar en paralelo con otras.\par

Los threads del mismo proceso comparten parte de su informacion: Memoria, variables globales, archivos abiertos, procesos hijos, temporizadores y semaforos. Que compartan la memoria genera que no haya proteccion de memoria y los threads son independientes unos de otros hasta cierto punto.\par
Pero no comparten: contador de programa, pila, registros y estado. En resumen, el contexto de ejecucion.\par

Las ventajas que proporciona este mecanismo de "dividir el trabajo de los procesos en tareas mas sencillas" es claro y sigue el paradigma de divide y venceras, sin embargo, cuando entramos en el terreno de la programacion concurrente hemos de ser cuidadosos por los tipicos problemas que pueden surgir.\par

\subsection{senales}

Se puede describir una senal como una interrupcion a un proceso. El comportamiento de las senales es similar al de las interrupciones en la cpu.\par

Las senales pueden provenir o bien de un proceso que la manda a otro proceso con su mismo \textit{uid} o el sistema operativo ante las tipicas excepciones que vemos cuando fallan nuestros programas (seg fault, /0...)\par

Hay muchas senales diferentes y se engloban en tres tipos fundamentales: De hardware, de comunicacion y de E/S asincrona.\par

Para que un proceso pueda enviar una senal ha de estar preparado para recibirla. Para ello tenemos que indicarle al SO el nombre de la rutina del proceso que ha de tratar ese tipo de senal. Si un proceso recibe una senal sin estar preparado para ello, se suele matar al proceso.

\subsection{Servicios de procesos}

Gestion de procesos:

\begin{itemize}
	\item{Identificador de proceso: PID (pid\_t)}
	\item{Obtener el identificador de proceso: getpid() devuelve el identificador del proceso que hizo la llamada}
	\item{Obtener el identificador del proceso padre: getppid()}
	\item{Obtener el identificador de usuario real: getuid()}
	\item{Obtener el identificador de usuario efectivo: geteuid()}
	\item{Obtener el identificador de grupo real: getgid()}
	\item{Obtener el identificador de grupo efectivo: getegid()}
	\item{Obtener el valor de una variable de entorno: *getenv( const char * name )}
\end{itemize}

\end{document}
